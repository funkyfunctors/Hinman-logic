\documentclass[a4paper, 11pt]{book}
\usepackage{comment} % enables the use of multi-line comments (\ifx \fi) 
\usepackage{lipsum} %This package just generates Lorem Ipsum filler text. 
\usepackage{fullpage} % changes the margin
\usepackage[a4paper, total={7in, 10in}]{geometry}

%\usepackage{tgadventor} % The font for the entire document can be changed here
%\usepackage{courier}
%\usepackage{charter}
\usepackage{tgcursor}

\usepackage{mathrsfs} 
\usepackage{quiver} 
\newtheorem{corollary}{Corollary}
\usepackage{graphicx}
\usepackage{tikz}

\usetikzlibrary{arrows}
\usepackage{verbatim}
\usepackage[numbered]{mcode}
\usepackage{float}
\usepackage{tikz}
    \usetikzlibrary{shapes,arrows}
    \usetikzlibrary{arrows,calc,positioning}

    \tikzset{
        block/.style = {draw, rectangle,
            minimum height=1cm,
            minimum width=1.5cm},
        input/.style = {coordinate,node distance=1cm},
        output/.style = {coordinate,node distance=4cm},
        arrow/.style={draw, -latex,node distance=2cm},
        pinstyle/.style = {pin edge={latex-, black,node distance=2cm}},
        sum/.style = {draw, circle, node distance=1cm},
    }
\usepackage{xcolor}
\usepackage{mdframed}
\usepackage[shortlabels]{enumitem}
\usepackage{indentfirst}
\usepackage{hyperref}
\usepackage{amsmath,amsfonts,amsthm, amssymb}
\usepackage{array}
\usepackage[all,textures]{xy}
\usepackage{graphicx}
\usepackage{alltt}
\usepackage{listings}
\usepackage{float}
\usepackage{tabu}
\usepackage{longtable}


\theoremstyle{plain}

\newtheorem{exercise}{Exercise}[chapter]
\newtheorem*{theorem1}{Theorem 1}
\newtheorem*{notation}{Notation}
\newtheorem*{corollary1.1}{Corollary 1.1}
\newtheorem*{theorem2}{Theorem 2}
\newtheorem*{corollary2.1}{Corollary 2.1}
\renewcommand{\thesubsection}{\thesection.\alph{subsection}}


% Define solution environment
\newenvironment{answer}
    {\textit{answer:}}
    {}
%%%%%%%%%%%%%%%%%%%%%%%%%%%%%%%%%%%%%%%%%%%%%%%%%%%%%%%%%%%%%%%%%%%%%%%%%%%%%%%%%%%%%%%%%%%%%%%%%%%%%%%%%%%%%%%%%%%%%%%%%%%%%%%%%%%%%%%% Original packages, custom environments, and custom commands below
  \usepackage{amsmath,amsfonts,amsthm, amssymb}
  \usepackage{fullpage}
  \usepackage{array}
  \usepackage[all,textures]{xy}
  \usepackage{graphicx}
  \usepackage{alltt}
  \usepackage{listings}
  \usepackage{float}
  \usepackage{tabu}
  \usepackage{longtable}
  \usepackage{lipsum}
  \usepackage[T1]{fontenc}
  
  \theoremstyle{plain}
  \newtheorem{innercustomgeneric}{\customgenericname}
\providecommand{\customgenericname}{}
\newcommand{\newcustomtheorem}[2]{%
  \newenvironment{#1}[1]
  {%
   \renewcommand\customgenericname{#2}%
   \renewcommand\theinnercustomgeneric{##1}%
   \innercustomgeneric
  }
  {\endinnercustomgeneric}
}

\newcustomtheorem{definition}{Definition}
\newcustomtheorem{lemma}{Lemma}

\newcommand{\mb}{\mathbf}
\newcommand{\arr}{\rightarrow}
\newcommand{\mc}{\mathcal}
\newcommand{\ms}{\mathscr}
\newcommand{\co}{\text{co}}
\newcommand{\N}{\mathbb{N}}
\newcommand{\Z}{\mathbb{Z}}
\newcommand{\Q}{\mathbb{Q}}
\newcommand{\R}{\mathbb{R}}
\newcommand{\I}{\mathbb{I}}
\newcommand{\la}{\langle}
\newcommand{\ra}{\rangle}
\newcommand{\op}{\oplus}
\newcommand{\od}{\odot}
\newcommand{\sent}{\text{Sent}}
\newcommand{\n}{\neg}
\newcommand{\bt}{\bullet}
\newcommand{\lrarr}{\leftrightarrow}


  \setlength{\parindent}{0pt}
 
 
 \newtheorem{thm}{Exercise}

\begin{document}

  \begin{titlepage}
	\centering % Center everything on the title page
	\scshape % Use small caps for all text on the title page
	\vspace*{1.5\baselineskip} % White space at the top of the page
% ===================
%	Title Section 	
% ===================

	\rule{13cm}{1.6pt}\vspace*{-\baselineskip}\vspace*{2pt} % Thick horizontal rule
	\rule{13cm}{0.4pt} % Thin horizontal rule
	
		\vspace{0.75\baselineskip} % Whitespace above the title
% ========== Title ===============	
	{	\Huge Fundamentals of Mathematical Logic \\ 
			\vspace{4mm}
      By Peter G. Hinman \\	}
% ======================================
		\vspace{0.75\baselineskip} % Whitespace below the title
	\rule{13cm}{0.4pt}\vspace*{-\baselineskip}\vspace{3.2pt} % Thin horizontal rule
	\rule{13cm}{1.6pt} % Thick horizontal rule
	
		\vspace{1.75\baselineskip} % Whitespace after the title block
% =================
%	Information	
% =================
	{\large : 
		\vspace*{1.2\baselineskip}
	} \\
	\vfill

\end{titlepage}
%%%%%%%%%%%%%%%%%%%%%%%%%%%%%%%%%%%%%%%%%%%%%%%%%%%%%%%%%%%
\begingroup
\let\cleardoublepage\clearpage
\tableofcontents
\endgroup

\chapter{Propositional Logic and Other Fundamentals}

  \begin{exercise}
    (Section I, 12)
    Give a careful proof of Proposition 1.1.9, and show how theorem 1.1.7 is an application of it. \\

    \textbf{Proposition 1.1.9: Sentence recursion theorem.} For any set $Z$, any function $F_0: \sent_0 \arr Z$ and any functions $G_\n: Z \arr Z$ and $G_\bt: Z \times Z \arr Z$, for $\bt $ any of $\lor$, $\land$, $\arr$, $\lrarr $, there exists a unique function $F: \sent_L \arr Z$ such that $F$ extends $F_0$ and for all $L$-senteces $\phi$ and $\psi$, $$F( \n \phi)= G_\n (F(\phi)) \text{ and } F(\bt \phi \psi)=G_\bt (F(\phi), F(\psi)).$$
  \end{exercise}
  \begin{proof}
    We shall define recursively a sequence of functions $F_n: \sent_n \arr Z$ for $n \in \omega$ such that for each $n$, $F_{n+1}$ is extension of $F_n$ and respect the desired relationship. Indeed, given such a $F_n$, we simply use the unqiue readability for propositional sentences to define values of $F_{n+1}$ on the members of $\sent_{n+1}$. Thus, for $\phi \in \sent_{n+1}$, set 
    $$ F_{n+1}(\phi) = 
      \begin{cases}
      F_n(\phi), & \text{if } \phi \in \sent_n \\
      G_\n (F_{n+1} (\psi)), & \text{if } \phi= \neg \psi \text{ where } \psi \in \sent_n \\
      G_\bt (F_{n+1}(\psi), F_{n+1}(\gamma)), & \text{if } \phi= \bt \psi \gamma \text{ where } \psi, \gamma \in \sent_n
      \end{cases} $$

    Now we define $F: \sent_L \arr Z$ by $F(\theta)=F_n(\theta)$ for some $n$ such that $\theta \in \sent_n$. Clearly, $F$ extends $F_0$ and satisfy the desired relationship because all $F_n$ do. Furthermore, unique extension of atomic truth assignemnt is a special case where $Z=\{T, F\}$, and $G_\n$ and $G_\bt$ are truth assignemnt functions.
  \end{proof}


\end{document}

